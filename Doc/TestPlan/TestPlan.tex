\documentclass[12pt, titlepage]{article}
\usepackage{booktabs}
\usepackage{tabularx}
\usepackage{hyperref}

\hypersetup{
	colorlinks,
	citecolor=black,
	filecolor=black,
	linkcolor=black,
	urlcolor=blue
}
\usepackage[round]{natbib}
\title{SE 3XA3: Test Plan\\PyCards}
\author{Team 2
	\\ Aravi Premachandran  premaa
	\\ Michael Lee  leemr2
	\\ Nikhil Patel  patelna2
}
\date{\today}

\begin{document}
	\maketitle
	\pagenumbering{roman}
	\tableofcontents
	\listoftables
	\listoffigures
	\begin{table}[bp]
		\caption{\bf Revision History}
		\begin{tabularx}{\textwidth}{p{3cm}p{2cm}X}
			\toprule {\bf Date} & {\bf Version} & {\bf Notes}\\
			\midrule
			October 27 & 0.1 & Rough Draft\\
			\bottomrule
		\end{tabularx}
	\end{table}
	\newpage
	\pagenumbering{arabic}
	\section{General Information}
	\subsection{Purpose}
	\indent \indent The purpose of conducting testing is twofold. While testing 
	cannot prove correctness or the absence of bugs, it can be useful for 
	finding instances of incorrect behaviour. By ensuring that testing is done 
	in a traceable and repeatable manner (namely through automation), defects 
	that are uncovered can be traced, isolated, and addressed. With automation, 
	testing can be performed throughout the life cycle of the product with very 
	little overhead.\par
	The other reason for conducting testing is to demonstrate to the 
	client	that our product is reliable, robust and meets the requirements 
	that were set forth (using fit criterion or some other measure of degree
	of fulfilment).
	\subsection{Scope}
	\indent\indent PyCards is a collection of card games implemented as a 
	desktop application. As with any software program, it is important that it 
	undergoes various iterations of testing throughout its product lifecyle.
	Our development team is using a number of different test types, including 
	functional, structural, and unit tests, static and dynamic, manual as well 
	as automated.
	While automated testing is largely preferred for reasons such as greater 
	traceability, reproducibility, and efficiency, testing will also need to be 
	done manually, especially for validating non-functional requirements.
	Thus, the scope of testing for this product includes functional, 
	structural, and unit tests, static and dynamic testing, and manual and 
	automated testing.
	
	\subsection{Acronyms, Abbreviations, and Symbols}
	
	\begin{table}[hbp]
		\caption{\textbf{Table of Abbreviations}} \label{AbbrTable}
		\begin{tabularx}{\textwidth}{p{3cm}X}
			\toprule
			\textbf{Abbreviation} & \textbf{Definition} \\
			\midrule
			VC++ 2008 \newline binaries & Required dlls included in the
			Microsoft Visual C++ 2008 redistributable package\\
			\bottomrule
		\end{tabularx}
	\end{table}
	\begin{table}[!htbp]
		\caption{\textbf{Table of Definitions}} \label{DefTable}
		\begin{tabularx}{\textwidth}{p{3cm}X}
			\toprule
			\textbf{Term} & \textbf{Definition}\\
			\midrule
			Widget & A functional component of a graphical user interface. 
			Usually	consists of a visual component and a callback\\
			Callback & A method to be executed upon interaction with a widget\\
			Bijection & A one-to-one mapping of elements in one set to another\\
			\bottomrule
		\end{tabularx}
	\end{table}	
	\subsection{Overview of Document}
	\indent \indent This document provides a detailed description of the 
	testing our development team has deemed necessary for the software product. 
	The tests are categorized and subdivided based on the type of testing, the 
	scope of said categories, and the purpose and application of the tests (ie. 
	validating the fulfillment functional or non-functional requirements).\par
	This document is subject to revision throughout the expected life of the 
	product. It is not expected that many deletions or shrinking of the test 
	sets will occur; however, additional testing will likely be prescribed and 
	document as the product is developed and matures.
	
	\section{Plan}
	This section details the testing process prescribed for the software 
	product, including but not limited to the testing team, schedule, 
	techniques, and technologies.
	\subsection{Software Description}
	PyCards is a collection of card games implemented as a desktop application.
	The implementation is written in pure Python. The application is to be made
	available both as a standalone executable and in a source version.
	\subsection{Test Team}
	\begin{itemize}
		\itemsep0em
		\item Aravi Premachandran
		\item Michael Lee
		\item Nikhil Patel
	\end{itemize}
	\subsection{Automated Testing Approach}
	\indent \indent The testing team will be applying automated testing for a 
	subset of the structural and static tests. In particular, unit tests will 
	primarily be automated to increase reproducibility and efficiency, among 
	other factors. It should be noted that in automated testing, only the 
	execution and evaluation (of pre-defined criteria) is automated - in the 
	event of failures or unexpected behaviour a member of the testing team will 
	still be required to analyze the requirements, the code, and the test 
	itself to determine where the inconsistency, if any, is located.
	
	\subsection{Testing Tools}
	\begin{itemize}
		\itemsep0em
		\item IDE - PyCharm
		\vspace{-3mm}
		\begin{itemize}
			\item static, structural: syntax checking, reachability, adherence 
			to coding conventions
		\end{itemize}
		\item pylint / pyCheckers
		\vspace{-3mm}
		\begin{itemize}
			\item static, stuctural: syntax checking, reachability, adherence 
			to coding conventions
		\end{itemize}
		\item coverage.py
		\vspace{-3mm}
		\begin{itemize}
			\item dynamic, stuctural: code coverage of the program
		\end{itemize}
		\item unittest.py - built-in module for testing
		\vspace{-3mm}
		\begin{itemize}
			\item dynamic, unit test
			\vspace{-2mm}
			\begin{itemize}
				\item mock module - can be used for stubs and drivers, to 
				isolate code
			\end{itemize}
		\end{itemize}
	\end{itemize}
		
	\subsection{Testing Schedule}
	
	See Gantt Chart at the following url ...
	\section{System Test Description}
	
	\subsection{Tests for Functional Requirements}
	\subsubsection{Event Handling}
	
	\paragraph{Key Bindings}
	\begin{enumerate}
		\item{KB1\\}
		Type: Functional, Dynamic, Manual
		
		Initial State: Application instance that is capturing user input
		
		Input: Keyboard press of one of COMMAND\_KEYS by user
		
		Output: Application perform the associated action from BOUND\_ACTIONS. 
		Please refer to \textbf{\autoref{keymap}} \nameref{keymap} for mapping of keys
		to actions
		
		How test will be performed: The application will be launched and once loaded,
		each bound key in COMMAND\_KEYS will be pressed and we will observe whether
		the correct action from BOUND\_ACTIONS executes
		
	\end{enumerate}
	\subsubsection{Widget Callbacks}
	\begin{enumerate}
		\item{WC1\\}
		Type: Functional, Dynamic, Manual
		
		Initial State: Menubar widget waiting for user interaction
		
		Input: User clicks on a label in the menu bar
		
		Output: If the user clicks or hovers on a cascading menu it will expand
		to show all contained submenu labels. If the user clicks on a menu label
		it will perform its associated callback function
		
		How test will be performed: With full knowledge of what behaviour should
		result from clicking any of the menu labels, the user will choose a random
		subset of labels to click, and compare the expected and actual behaviour of
		the application

		\item{WC2\\}
		Type: Functional, Dynamic, Manual
		
		Initial State: Toolbar widget waiting for user interaction
		
		Input: User clicks on a toolbar button
		
		Output: The application will execute the appropriate function based upon
		the toolbar button clicked. There is a bijection between the actions in
		BOUND\_ACTIONS and toolbar buttons
		
		How test will be performed: The application will be launched and once loaded,
		each bound key in COMMAND\_KEYS will be pressed and we will observe whether
		the correct action from BOUND\_ACTIONS executes

		\item{WC3\\}
		Type: Functional, Dynamic, Manual
		
		Initial State: Card widgets waiting for user click
		
		Input: Primary button click on card widget
		
		Output: If the card selection is valid (see Functional Klondike Requirements in 
		\href{https://gitlab.cas.mcmaster.ca/premaa/pysol/tree/master/Doc/SRS}
		{PyCards SRS} for selection constraints) the card is highlighted while the
		mouse button remains pressed and is redrawn to follow the cursor. If the card
		selection is invalid no visible changes are made to the window.
		
		How test will be performed: With a game in progress two cards will be clicked
		one 'valid' and the other an 'invalid' selection and we will observe to make
		sure only the 'valid' selection results in a card being highlighted	
	\end{enumerate}

	\subsection{Tests for Nonfunctional Requirements}
	\subsubsection{Interoperability}
	
	\paragraph{Operating System Compatibility}
	\begin{enumerate}
		\item{OSX\\}
		Type: Structural, Dynamic, Manual
		
		Initial State: Executable for program is on target machine running a
		standard install of OS X OSX\_VERSION.
		
		Input/Condition: User locates executable and launches program
		
		Output/Result: Either a successful launch or a message from Gatekeeper
		alerting user that program was created by an unidentified developer
		
		How test will be performed: Executable is launched via double clicking to
		test whether or not Gatekeeper feature will block normal launch. User will
		observe whether basic functionality such as input/output and rendering of
		application window are successful
		
		\item{OS2\\}
		Type: Structural, Dynamic, Manual
		
		Initial State: Executable for program is on target machine running a
		standard install of Ubuntu UBUNTU\_VERSION.
		
		Input/Condition: User locates executable and launches program
		
		Output/Result: Either a successful launch or a failure caused by incompatible
		packaging, dependencies, or other causes
		
		How test will be performed: Executable is located and then launched via
		terminal on the target system. User will observe whether basic
		functionality such as input/output and rendering of application window are
		successful

		\item{OSWIN\\}
		Type: Structural, Dynamic, Manual
		
		Initial State: Executable for program is on target machine running a
		standard install of Windows WIN\_VERSION.
		
		Input/Condition: User locates executable and launches program via double-click
		
		Output/Result: Depending on the system, either a successful launch or
		application crash due to missing VC++2008 binaries (required even after 
		building executable)
		
		How test will be performed: Executable is located and then launched via
		double-click on the target system. User will observe whether basic
		functionality such as input/output and rendering of application window are
		successful
	\end{enumerate}

	\paragraph{Portability}
	\begin{enumerate}
		\item{PR1\\}
		Type: Structural, Dynamic, Manual
		
		Initial State: Executable for application is located on a removable USB drive
		
		Input/Condition: User launches program executable from a removable USB drive
		
		Output/Result: The program launches exactly as if it had been launched on
		the same system from the internal hard drive
		
		How test will be performed: User inserts removable drive into a machine running a
		supported OS then executes the executable either via double-clicking or using
		a shell (ie. bash, command-line). User will observe whether or not the program
		successfuly launches and operates

		\item{PR2\\}
		Type: Structural, Static, Manual
		
		Initial State: The VC++ 2008 binaries required for execution of the program
		are not found on the host system
		
		Input: The program fails to launch on a Windows-based machine that does not
		have the VC++ 2008 redistributable package installed
		
		Output: All of the users are able to successfully follow the provided
		instructions to install the VC++ 2008 redistributable package

		How test will be performed: A group of randomly selected users are given 
		this issue and will be prompted by our application with instructions for
		resolving it. The criteria for success of this test is that all users are able
		to follow the instructions, install the required package, and after doing
		so successfully launch the application \newline
		Tolerance and Limitations: This test is limited to being performed on
		machines running a Windows installation with minimum WIN\_VERSION.
		If the success criteria is not met because of unforeseen errors users will
		be advised to download and run the application from source
	\end{enumerate}

	\subsubsection{Usability}
	\paragraph{Navigation}
	\begin{enumerate}
	\item{UN1\\}
		Type: Structural, Dynamic, Manual
		
		Initial State: The existing implementation is running and idle on the user's
		machine, and is waiting for user interaction
		
		Input/Condition: A group of users are asked to perform each of the actions
		defined in BOUND\_ACTIONS
		
		Output/Result: The majority of users successfully perform the different
		actions, completing each within in a period of under MAX\_FIND\_TIME
		
		How test will be performed: A group of randomly chosen people, of different
		ages and unspecificied level of technical expertise will be asked to perform
		the actions defined in BOUND\_ACTIONS. The amount of time it takes them
		to perform each task will be measured. The test will be considered successfull
		if the majority of users are able to complete each of the individual actions
		in under MAX\_FIND\_TIME	

		\item{UN2\\}
		Type: Structural, Dynamic, Manual
		
		Initial State: Users have completed and taken note of the results of the
		previous test. Our application, PyCards, is now running and idle, waiting for
		user interaction
		
		Input/Condition: The group of users are asked to perform each of the actions
		defined in BOUND\_ACTIONS
		
		Output/Result: The majority of users successfully perform the different
		actions, completing each within in a period of under MAX\_FIND\_TIME and in
		time less than or equal to what was required to perform the same action using
		the existing implementation
		
		How test will be performed: A group of randomly chosen people, of different
		ages and unspecificied level of technical expertise will be asked to perform
		the actions defined in BOUND\_ACTIONS. The amount of time it takes them
		to perform each task will be measured. The test will be considered successfull
		if the majority of users are able to complete each of the individual actions
		in under MAX\_FIND\_TIME and in time less than or equal to time taken using
		the existing implementation	
	\end{enumerate}

	\paragraph{Playability}
	\begin{enumerate}
		\item{UP1\\}
		Type: Structural, Dynamic, Manual
		
		Initial State: Existing implementation is running, with a game in progress
		
		Input/Condition: A group of users are asked to play a game and rate it based
		ease of use using a scale from 1-5 where 5 is very user friendly
		
		Output/Result: The majority of users give the game an average rating above 3
		
		How test will be performed: A group of randomly chosen people, of different
		ages and unspecificied level of technical expertise will be asked the
		questions defined in the Appendix and to give answers
		from a scale from 1-5. The scale maps as: 1 - Very hard to use, 2 - Hard to
		use, 3 - Some effort to use, 4 - Easy to use, 5 - Very easy to use	

		\item{UP2\\}
		Type: Structural, Dynamic, Manual
		
		Initial State: Users have completed the previous test. PyCards is now running,
		with a game in progress
		
		Input/Condition: A group of users are asked to play the game and rate it based
		ease of use using a scale from 1-5 where 5 is very user friendly
		
		Output/Result: The majority of users give the game an average rating above 3
		and greater than or equal to the rating given in the previous test
		
		How test will be performed: A group of randomly chosen people, of different
		ages and unspecificied level of technical expertise will be asked the
		questions defined in the Appendix and to give answers
		from a scale from 1-5. The scale maps as: 1 - Very hard to use, 2 - Hard to
		use, 3 - Some effort to use, 4 - Easy to use, 5 - Very easy to use		
	\end{enumerate}

	\subsubsection{Product Integrity}
	\paragraph{Resource Loading}
	\begin{enumerate}
		\item{RL1\\}
		Type: Structural, Dynamic, Manual
		
		Initial State: The 'cardsets' directory is missing or corrupt
		
		Input/Condition: When loading the application prompts users to either
		re-download the application or download specifically the cardsets directory
		
		Output/Result: The majority of users are able to find the repository for the
		program and re-download the executable or the cardset directory
		
		How test will be performed: A group of users selected and willing to
		participate in this test scenario will be provided with a modified executable
		or the program's source excluding the cardsets directory. Upon launching the
		application will attempt to load the directory and find it missing. The
		application must not crash but instead prompt the user to redownload it
		
		\item{RL2\\}
		Type: Structural, Dynamic, Manual
		
		Initial State: The 'tiles' directory is missing or corrupt
		
		Input/Condition: Application attempts to load images from the 'tiles' directory
		
		Output/Result: An exception is thrown and handled by using a solid color tile
		as the window background.
		
		How test will be performed: Before launching the application the 'tiles'
		directory will be removed from the host system. Success for this test is
		defined by the application drawing a solid color background and continuing
		to function normally in all other regards
	\end{enumerate}
	
	\subsubsection{Security}
	\paragraph{System Permissions}
	\begin{enumerate}
		\item{SP1\\}
		Type: Structural, Dynamic, Manual
		
		Initial State: Application is located on the host machine's file system
		
		Input/Condition: User launches the application
		
		Output/Result: The application should launch without asking for administrative
		(root) privileges.
		
		How test will be performed: On Ubuntu, the application will be launched via
		terminal. It must launch without asking for superuser (root) permissions
		\newline On Windows, the application should launch successfully when double-
		clicked, not requiring the user to right-click and select 'Run as 
		administrator'
		\newline Tolerance: OS X implements a functionality called Gatekeeper that
		restricts the user from opening applications from unidentified developers. 
		This is a security protocol and our app will not attempt to circumvent it.
		Instead, the user should be able to control-click the app to launch it, after
		confirming their intention to bypass Gatekeeper
	\end{enumerate}

	\section{Tests for Proof of Concept}
	\subsection{Game Logic}
	
	\paragraph{Card Selection}
	\begin{enumerate}
		\item{PC1\\}
		Type: Structural, Dynamic, Automated
		
		Initial State: Klondike game is loaded
		
		Input: User selects card with mouse click
		
		Output: Clicked card is highlighted and tracks mouse cursor if valid 
		selection. If invalid selection the click is ignored. Criteria for valid 
		selection is defined in the SRS 
		
		How test will be performed: A random sequence of cards will be
		programmatically clicked to simulate user clicks. The criteria for valid 
		selection will be evaluated and the invocation of the correct method will be
		the success criteria for this test

		\item{PC2\\}
		Type: Structural, Static, Manual
		
		Initial State: The rules of the game written using first-order logic
		
		Input: A sequence of possible executions that covers the different conditions
		
		Output: Truth value whether the result of the execution conforms to the rules
		of the game
		
		How test will be performed: The logic of the game will be mapped as a set of
		states and transitions, in a control flow diagram. Each possible execution
		inputted represents a different path. The truth value of the execution is
		whether or not the rules of the game accept the trace
	\end{enumerate} 

	\section{Comparison to Existing Implementation}
		A subset of our tests compares the functionality of our program to that of the
		existing implementation. These tests include : 
	\begin{itemize}
		\itemsep0em
		\item test PC1 in Tests for Proof of Concept
		\item test PC2 in Tests for Proof of Concept
		\item test UN2 in Nonfunctional Requirements - Usability
		\item test UP2 in Nonfunctional Requirements - Usability
	\end{itemize}
	Tests PC1 and PC2 assert that the functionality of PyCards is analogous to the
	functionality of the existing implementation in terms of enforcing game rules
	\newline Tests UN2 and UP2 asserts that the design of PyCards is just as, if
	not more, user-friendly than that of the existing implementation
		
	
	\section{Unit Testing Plan}
	The unittest.py module will be used for automated unit testing for this
	project and the coverage.py module will be used to analyze code coverage

	\subsection{Unit testing of internal functions}
	Unit tests are used to validate the intermal components such as modules, 
	methods, and functions. In order to do this we model the compent as a series of
	actions and transitions or by using first order logic depending on the nature of
	the statements being executed. This behaviour can then be formalized by
	modelling it as a control flow graph or truth table to assert that the
	behaviour of the component conforms to the requirements placed upon it.
	For example, in the data-related logic these requirements can also be
	specified in some circumstances as preconditions and postconditions \newline
	Automating the testing using unittest.py enables us to make the tests
	repeatable and therefore provide a large series of inputs and outputs that
	model normal behaviour, edge cases and boundary conditions. The testing is also
	dynamic and as such we can also write helper methods to validate the state of
	objects at the beginning and end of tests
	
	\subsection{Unit testing of output files}	
	Our application is a game, and as such the output is visual in the form of the
	graphical user interface. This will be tested through our nonfunctional
	requirements such as usability and playability.

	\bibliographystyle{plainnat}
	\bibliography{TestPlan}
	\newpage
	\section{Appendix}

	\subsection{Symbolic Parameters}
	\begin{itemize}
			\itemsep0em
			\item COMMAND\_KEYS = \{'N', 'P', 'U', 'R', 'D', 'H', \newline 'F1', 'F2', CTRL+N,
			CTRL+P, CTRL+Q\}
			\item BOUND\_ACTIONS = \{newgame, pause, undo, redo, deal, help, quit\}
			\item MAX\_FIND\_TIME = 90 seconds
			\item NUM\_REDEALS = Unlimited
			\item FOUNDATION\_BASE = Ace
			\item STACK\_BASE = King
			\item NUM\_DECKS = 1
			\item NUM\_WASTES = 1
			\item NUM\_FOUNDATIONS = NUM\_SUITS = 4
			\item NUM\_NORMAL\_STACKS = 7
			\item OSX\_VERSION = EL\_CAPITAN
			\item UBUNTU\_VERSION = 16.04.1
			\item WIN\_VERSION = 10
	\end{itemize}

	\subsection{Reference Tables}
	\begin{table}[hbp]
		\caption{\textbf{Keymapping Configurations}} \label{keymap}
		\begin{tabularx}{\textwidth}{p{6cm}X}
			\toprule
			\textbf{COMMAND\_KEYS} & \textbf{BOUND\_ACTIONS} \\
			\midrule
			'CTRL+N', 'N', 'F2' & new\_game\\
			'CTRL+P', 'P' & pause\_game\\
			'CTRL+Q', 'Q' & quit\\
			'CTRL+H', 'H', 'F1' & show\_help\\
			'U' & undo\_move\\
			'R' & redo\_move\\
			'D' & deal\_card\\
			\bottomrule
		\end{tabularx}
	\end{table}

	\subsection{Usability Survey Questions}
	\begin{itemize}
		\itemsep0em
		\item How would you rate the user interface?
		\item How user-friendly were the in-game dialogs?
		\item How would you rate the overall game experience?
	\end{itemize}
\end{document}