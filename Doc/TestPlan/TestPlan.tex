\documentclass[12pt, titlepage]{article}
\usepackage{booktabs}
\usepackage{tabularx}
\usepackage{hyperref}
\hypersetup{
	colorlinks,
	citecolor=black,
	filecolor=black,
	linkcolor=black,
	urlcolor=blue
}
\usepackage[round]{natbib}
\title{SE 3XA3: Test Plan\\PyCards}
\author{Team 2
	\\ Aravi Premachandran  premaa
	\\ Michael Lee  leemr2
	\\ Nikhil Patel  patelna2
}
\date{\today}

\begin{document}
	\maketitle
	\pagenumbering{roman}
	\tableofcontents
	\listoftables
	\listoffigures
	\begin{table}[bp]
		\caption{\bf Revision History}
		\begin{tabularx}{\textwidth}{p{3cm}p{2cm}X}
			\toprule {\bf Date} & {\bf Version} & {\bf Notes}\\
			\midrule
			October 27 & 0.1 & Rough Draft\\
			\bottomrule
		\end{tabularx}
	\end{table}
	\newpage
	\pagenumbering{arabic}
	\section{General Information}
	\subsection{Purpose}
	\indent \indent The purpose of conducting testing is twofold. While testing 
	cannot prove correctness or the absence of bugs, it can be useful for 
	finding instances of incorrect behaviour. By ensuring that testing is done 
	in a traceable and repeatable manner (namely through automation), defects 
	that are uncovered can be traced, isolated, and addressed. With automation, 
	testing can be performed throughout the life cycle of the product with very 
	little overhead.\par
	The other reason for conducting testing is to demonstrate to the 
	client	that our product is reliable, robust and meets the requirements 
	that were set forth (using fit criterion or some other measure of degree
	of fulfilment).
	\subsection{Scope}
	\indent\indent PyCards is a collection of card games implemented as a 
	desktop application. As with any software program, it is important that it 
	undergoes various iterations of testing throughout its product lifecyle.
	Our development team is using a number of different test types, including 
	functional, structural, and unit tests, static and dynamic, manual as well 
	as automated.
	While automated testing is largely preferred for reasons such as greater 
	traceability, reproducibility, and efficiency, testing will also need to be 
	done manually, especially for validating non-functional requirements.
	Thus, the scope of testing for this product includes functional, 
	structural, and unit tests, static and dynamic testing, and manual and 
	automated testing.
	
	\subsection{Acronyms, Abbreviations, and Symbols}
	
	\begin{table}[hbp]
		\caption{\textbf{Table of Abbreviations}} \label{Table}
		\begin{tabularx}{\textwidth}{p{3cm}X}
			\toprule
			\textbf{Abbreviation} & \textbf{Definition} \\
			\midrule
			Abbreviation1 & Definition1\\
			Abbreviation2 & Definition2\\
			\bottomrule
		\end{tabularx}
	\end{table}
	\begin{table}[!htbp]
		\caption{\textbf{Table of Definitions}} \label{Table}
		\begin{tabularx}{\textwidth}{p{3cm}X}
			\toprule
			\textbf{Term} & \textbf{Definition}\\
			\midrule
			Term1 & Definition1\\
			Term2 & Definition2\\
			\bottomrule
		\end{tabularx}
	\end{table}	
	\subsection{Overview of Document}
	\indent \indent This document provides a detailed description of the 
	testing our development team has deemed necessary for the software product. 
	The tests are categorized and subdivided based on the type of testing, the 
	scope of said categories, and the purpose and application of the tests (ie. 
	validating the fulfillment functional or non-functional requirements).\par
	This document is subject to revision throughout the expected life of the 
	product. It is not expected that many deletions or shrinking of the test 
	sets will occur; however, additional testing will likely be prescribed and 
	document as the product is developed and matures.
	
	\section{Plan}
	This section details the testing process prescribed for the software 
	product, including but not limited to the testing team, schedule, 
	techniques, and technologies.
	\subsection{Software Description}

	\subsection{Test Team}
	\begin{itemize}
		\itemsep0em
		\item Aravi Premachandran
		\item Michael Lee
		\item Nikhil Patel
	\end{itemize}
	\subsection{Automated Testing Approach}
	\indent \indent The testing team will be applying automated testing for a 
	subset of the structural and static tests. In particular, unit tests will 
	primarily be automated to increase reproducibility and efficiency, among 
	other factors. It should be noted that in automated testing, only the 
	execution and evaluation (of pre-defined criteria) is automated - in the 
	event of failures or unexpected behaviour a member of the testing team will 
	still be required to analyze the requirements, the code, and the test 
	itself to determine where the inconsistency, if any, is located.
	
	\subsection{Testing Tools}
	\begin{itemize}
		\itemsep0em
		\item IDE - PyCharm
		\vspace{-3mm}
		\begin{itemize}
			\item static, structural: syntax checking, reachability, adherence 
			to coding conventions
		\end{itemize}
		\item pylint / pyCheckers
		\vspace{-3mm}
		\begin{itemize}
			\item static, stuctural: syntax checking, reachability, adherence 
			to coding conventions
		\end{itemize}
		\item unittest.py - built-in module for testing
		\vspace{-3mm}
		\begin{itemize}
			\item dynamic, unit test
			\vspace{-2mm}
			\begin{itemize}
				\item mock module - can be used for stubs and drivers, to 
				isolate code
			\end{itemize}
		\end{itemize}
	\end{itemize}
		
	\subsection{Testing Schedule}
	
	See Gantt Chart at the following url ...
	\section{System Test Description}
	
	\subsection{Tests for Functional Requirements}
	\subsubsection{Area of Testing1}
	
	\paragraph{Title for Test}
	\begin{enumerate}
		\item{test-id1\\}
		Type: Functional, Dynamic, Manual, Static etc.
		
		Initial State: 
		
		Input: 
		
		Output: 
		
		How test will be performed: 
		
		\item{test-id2\\}
		Type: Functional, Dynamic, Manual, Static etc.
		
		Initial State: 
		
		Input: 
		
		Output: 
		
		How test will be performed: 
	\end{enumerate}
	\subsubsection{Area of Testing2}
	...
	\subsection{Tests for Nonfunctional Requirements}
	\subsubsection{Area of Testing1}
	
	\paragraph{Title for Test}
	\begin{enumerate}
		\item{test-id1\\}
		Type: 
		
		Initial State: 
		
		Input/Condition: 
		
		Output/Result: 
		
		How test will be performed: 
		
		\item{test-id2\\}
		Type: Functional, Dynamic, Manual, Static etc.
		
		Initial State: 
		
		Input: 
		
		Output: 
		
		How test will be performed: 
	\end{enumerate}
	\subsubsection{Area of Testing2}
	...
	\section{Tests for Proof of Concept}
	\subsection{Area of Testing1}
	
	\paragraph{Title for Test}
	\begin{enumerate}
		\item{test-id1\\}
		Type: Functional, Dynamic, Manual, Static etc.
		
		Initial State: 
		
		Input: 
		
		Output: 
		
		How test will be performed: 
		
		\item{test-id2\\}
		Type: Functional, Dynamic, Manual, Static etc.
		
		Initial State: 
		
		Input: 
		
		Output: 
		
		How test will be performed: 
	\end{enumerate}
	\subsection{Area of Testing2}
	...
	
	\section{Comparison to Existing Implementation}	
	
	\section{Unit Testing Plan}
	
	\subsection{Unit testing of internal functions}
	
	\subsection{Unit testing of output files}		
	\bibliographystyle{plainnat}
	\bibliography{SRS}
	\newpage
	\section{Appendix}
	This is where you can place additional information.
	\subsection{Symbolic Parameters}
	The definition of the test cases will call for SYMBOLIC\_CONSTANTS.
	Their values are defined in this section for easy maintenance.
	\subsection{Usability Survey Questions?}
	This is a section that would be appropriate for some teams.
\end{document}