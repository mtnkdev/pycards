\documentclass[12pt, titlepage]{article}

\usepackage{booktabs}
\usepackage{tabularx}
\usepackage{hyperref}
\hypersetup{
    colorlinks,
    citecolor=black,
    filecolor=black,
    linkcolor=red,
    urlcolor=blue
}
\usepackage[round]{natbib}
\newcommand{\tref}[1]{T\ref{#1}}

\title{SE 3XA3: Test Report\\PyCards}

\author{Team 2
		\\ Aravi Premachandran premaa
		\\ Michael Lee  leemr2
		\\ Nikhil Patel  patelna2
}

\date{\today}

%\input{../Comments}

\begin{document}

\maketitle

\pagenumbering{roman}
\tableofcontents
\listoftables
\listoffigures

\begin{table}[bp]
\caption{\bf Revision History}
\begin{tabularx}{\textwidth}{p{3cm}p{2cm}X}
\toprule {\bf Date} & {\bf Version} & {\bf Notes}\\
\midrule
December 4 & 1.0 & Initial Revision\\
\bottomrule
\end{tabularx}
\end{table}

\newpage

\pagenumbering{arabic}


\section{Functional Requirements Evaluation}

	\subsection{Event Handling}
	
	\paragraph{Key Bindings (\tref{tKeys})}
	\begin{enumerate}
		\item{KB1\\}
		Type: Functional, Dynamic, Manual
		
		Initial State: Application instance that is capturing user input
		
		Input: Keyboard press of one of COMMAND\_KEYS by user
		
		Expected Results: The corect response from BOUND\_ACTIONS executes 

		Success/Failure: Success
		
	\end{enumerate}
	\subsection{Widget Callbacks (\tref{tClick})}
	\begin{enumerate}
		\item{WC1\\}
		Type: Functional, Dynamic, Manual
		
		Initial State: Menubar widget waiting for user interaction
		
		Input: User clicks on a label in the menu bar
		
		Expected Results: If the user clicks or hovers on a cascading menu it will expand
		to show all contained submenu labels. If the user clicks on a menu label
		it will perform its associated callback function.

		Success/Failure: Success

		\item{WC2\\}
		Type: Functional, Dynamic, Manual
		
		Initial State: Card widgets waiting for user click
		
		Input: Primary button click on card widget
		
		Expected Results: If the card selection is valid (see Functional Klondike Requirements in 
		\href{https://gitlab.cas.mcmaster.ca/premaa/pysol/tree/master/Doc/SRS}
		{PyCards SRS} for selection constraints) the card is highlighted while the
		mouse button remains pressed and is redrawn to follow the cursor. If the card
		selection is invalid no visible changes are made to the window.

		Success/Failure: Success
	\end{enumerate}
	\subsection{Game Logic}
	
	\paragraph{Card Selection (\tref{tCard})}
	\begin{enumerate}
		\item{PC1\\}
		Type: Structural, Dynamic, Automated
		
		Initial State: Klondike game is loaded
		
		Input: User selects card with mouse click
		
		Expected Results: Clicked card is highlighted and tracks mouse cursor if valid 
		selection. If invalid selection the click is ignored. Criteria for valid 
		selection is defined in the SRS

		Success/Failure: Success

		\item{PC2\\}
		Type: Structural, Static, Manual
		
		Initial State: The rules of the game written using first-order logic
		
		Input: A sequence of possible executions that covers the different conditions
		
		Expected Results: Truth value whether the result of the execution conforms to the rules
		of the game

		Success/Failure: Success
	\end{enumerate} 



\section{Nonfunctional Requirements Evaluation}

\subsection{Interoperability}
	
	\paragraph{Operating System Compatibility (\tref{tOS})}
	\begin{enumerate}
		\item{OSX\\}
		Type: Structural, Dynamic, Manual
		
		Initial State: Executable for program is on target machine running a
		standard install of OS X OSX\_VERSION.
		
		Input/Condition: User locates executable and launches program
		
		Expected Results: Either a successful launch or a message from Gatekeeper
		alerting user that program was created by an unidentified developer

		Success/Failure: Success
		
		\item{OS2\\}
		Type: Structural, Dynamic, Manual
		
		Initial State: Executable for program is on target machine running a
		standard install of Ubuntu UBUNTU\_VERSION.
		
		Input/Condition: User locates executable and launches program
		
		Expected Results: Either a successful launch or a failure caused by incompatible
		packaging, dependencies, or other causes

		Success/Failure: Success

		\item{OSWIN\\}
		Type: Structural, Dynamic, Manual
		
		Initial State: Executable for program is on target machine running a
		standard install of Windows WIN\_VERSION.
		
		Input/Condition: User locates executable and launches program via double-click
		
		Expected Results: Depending on the system, either a successful launch or
		application crash due to missing VC++2008 binaries (required even after 
		building executable)

		Success/Failure: Success
	\end{enumerate}

	\paragraph{Portability (\tref{tPort})}
	\begin{enumerate}
		\item{PR1\\}
		Type: Structural, Dynamic, Manual
		
		Initial State: Executable for application is located on a removable USB drive
		
		Input/Condition: User launches program executable from a removable USB drive
		
		Expected Results: The program launches exactly as if it had been launched on
		the same system from the internal hard drive

		Success/Failure: Success
	\end{enumerate}

	\subsection{Usability (\tref{tUse})}
	\paragraph{Navigation}
	\begin{enumerate}
	\item{UN1\\}
		Type: Structural, Dynamic, Manual
		
		Initial State: The existing implementation is running and idle on the user's
		machine, and is waiting for user interaction
		
		Input/Condition: A group of users are asked to perform each of the actions
		defined in BOUND\_ACTIONS
		
		Expected Results: The majority of users successfully perform the different
		actions, completing each within in a period of under MAX\_FIND\_TIME

		Success/Failure: Success

		\item{UN2\\}
		Type: Structural, Dynamic, Manual
		
		Initial State: Users have completed and taken note of the results of the
		previous test. Our application, PyCards, is now running and idle, waiting for
		user interaction
		
		Input/Condition: The group of users are asked to perform each of the actions
		defined in BOUND\_ACTIONS
		
		Expected Results: The majority of users successfully perform the different
		actions, completing each within in a period of under MAX\_FIND\_TIME and in
		time less than or equal to what was required to perform the same action using
		the existing implementation

		Success/Failure: Success
	\end{enumerate}

	\paragraph{Playability (\tref{tPlay})}
	\begin{enumerate}
		\item{UP1\\}
		Type: Structural, Dynamic, Manual
		
		Initial State: Existing implementation is running, with a game in progress
		
		Input/Condition: A group of users are asked to play a game and rate it based
		ease of use using a scale from 1-5 where 5 is very user friendly
		
		Expected Results: The majority of users give the game an average rating above 3

		Success/Failure: Success

		\item{UP2\\}
		Type: Structural, Dynamic, Manual
		
		Initial State: Users have completed the previous test. PyCards is now running,
		with a game in progress
		
		Input/Condition: A group of users are asked to play the game and rate it based
		ease of use using a scale from 1-5 where 5 is very user friendly
		
		Expected Results: The majority of users give the game an average rating above 3
		and greater than or equal to the rating given in the previous test

		Success/Failure: Success
	\end{enumerate}

	\subsection{Product Integrity (\tref{tInt})}
	\paragraph{Resource Loading}
	\begin{enumerate}
		\item{RL1\\}
		Type: Structural, Dynamic, Manual
		
		Initial State: The 'cardsets' directory is missing or corrupt
		
		Input/Condition: When loading the application prompts users to either
		re-download the application or download specifically the cardsets directory
		
		Expected Results: The majority of users are able to find the repository for the
		program and re-download the executable or the cardset directory

		Success/Failure: Success
		
		\item{RL2\\}
		Type: Structural, Dynamic, Manual
		
		Initial State: The 'tiles' directory is missing or corrupt
		
		Input/Condition: Application attempts to load images from the 'tiles' directory
		
		Expected Results: An exception is thrown and handled by using a solid color tile
		as the window background

		Success/Failure: Success
	\end{enumerate}
	
	\subsection{Security (\tref{tSec})}
	\paragraph{System Permissions}
	\begin{enumerate}
		\item{SP1\\}
		Type: Structural, Dynamic, Manual
		
		Initial State: Application is located on the host machine's file system
		
		Input/Condition: User launches the application
		
		Expected Results: The application should launch without asking for administrative
		(root) privileges

		Success/Failure: Success
	\end{enumerate}


\section{Unit Testing}

Unit testing was conducted using the unittest.py python module. Test suites
were constructed for the core modules that make up the controller and model
conceptual modules, as that is where the decision and data logic lies. More
specifically, unit testing was performed to validate the correct behaviour of
the dealer, mouse\_handler, assets, and game-containing modules. \newline
\indnet For each game the key behaviours that needs to be validated include the
creation of a game, the dealing of a game, the selection and movement of
cards, and the updates to a game (ie flipping of cards, movement of cards in
spider solitaire). These behaviours are tested by first asserting the necessary
preconditions, then executing the action, then asserting the postconditions.
If both the preconditions and postconditions are met then the test is considered
to have executed successfuly.

\section{Changes Due to Testing}

\section{Automated Testing}

Automated testing was done using the unittest.py python module. This allowed us to create test suites for the various methods and use assert statements to specify preconditions and postconditions.
		
\section{Trace to Requirements}

\begin{table}[h]
		\centering
		\begin{tabular}{p{0.2\textwidth} p{0.6\textwidth}}
			\toprule
			\textbf{Test Scenario} & \textbf{Requirements}\\
			\midrule
			KB1 & FR4, **GR1
			WC1 & GR1, FR4
			WC2 & GR1, FR4
			PC1 & GR3
			PC2 & GR3
			OSX & NFR4, NFR6, FR1
			OS2 & NFR4, NFR6, FR1
			OSWIN & NFR4, NFR6, FR1
			PR1 & FR1, NFR5
			UN1 & GR1, FR4
			UP1 & NFR1, NFR2, NFR3
			RL1 & FR2
			RL2 & FR2
			SP1 & FR1, FR2, NFR5, NFR6
			REQS & \tref{tClick}, \tref{tUse}, \tref{tInt}, \tref{tSec}\\
			REQS & \tref{tKeys}, \tref{tCard}, \tref{tOS}, \tref{tPort}\\
			REQS & \tref{tUse}\\
			\bottomrule
		\end{tabular}
		\caption{Trace Between Tests and Requirements}
		\label{TblRT}
	\end{table}
		
\section{Trace to Modules}		


\begin{table}[h]
		\centering
		\begin{tabular}{p{0.2\textwidth} p{0.6\textwidth}}
			\toprule
			\textbf{Test Scenario} & \textbf{Modules}\\
			\midrule
			KB1 & mKB
			WC1 & mMI
			WC2 & mMI
			PC1 & mCM, mGP
			PC2 & mCM, mGP
			OSX & mPH
			OS2 & mPH
			OSWIN & mPH
			PR1 & mPH
			UN1 & mGP, mUS
			UP1 & mGP, mUS
			RL1 & mGP, mUS
			RL2 & mGP, mUS
			SP1 & mUS, mSL
			%not needed REQS & \tref{tClick}, \tref{tUse}, \tref{tInt}, \tref{tSec}\\
			%not needed REQS & \tref{tKeys}, \tref{tCard}, \tref{tOS}, \tref{tPort}\\
			%not needed REQS & \tref{tUse}\\
			\bottomrule
		\end{tabular}
		\caption{Trace Between Tests and Modules}
		\label{TblTM}
	\end{table}


\section{Code Coverage Metrics}

Code coverage was done using the coverage.py python module and running it
against our project. The coverage.py module works in three phases: first it
executes our code and monitors the statements that were executed, then it
examines the source to determine which lines could have run, and finally it
provides a report in the desired format (ie. text, html, annotated source).

\bibliographystyle{plainnat}

\bibliography{SRS}

\end{document}