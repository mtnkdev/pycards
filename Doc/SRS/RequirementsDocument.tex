\documentclass{article}
\usepackage{setspace}
\usepackage[letterpaper, margin=1in]{geometry}
\usepackage{booktabs}
\usepackage{tabularx}
\usepackage{nameref}

\title{SE 3XA3: Software Requirements Specification\\PyCards}
\author{Team 2,
		\\ Aravi Premachandran  premaa
		\\ Michael Lee  leemr2
		\\ Nikhil Patel  patelna2
}
\date{October 11, 2016}

\newcommand {\PYVER}{2.7.xx }


\begin{document}
	\maketitle

	\newpage
	\tableofcontents
	\listoftables
	\listoffigures

	\begin{table}[tp]
		\caption{\bf Revision History}
		\begin{tabularx}{\textwidth}{p{3cm}p{2cm}X}
			\toprule {\bf Date} & {\bf Version} & {\bf Notes}\\
			\midrule
			October 11, 2016 & 1.0 & Initial Revision\\
			\bottomrule
		\end{tabularx}
	\end{table}

	\newpage
	\section{Project Drivers}
		\subsection{The Purpose of the Project}
		\indent \indent PyCards is a collection of solitaire 
		(single-player) card games. It is designed to be a source of 
		entertainment for the end 
		user(s). The main objective for the product is to PyCards to be a 
		user-friendly application that our users can use to pass the time, have 
		fun, relax, and also challenge themselves.
	\subsection{The Stakeholders}
		The Client\\
		The Customers\\
		Other Stakeholders
	\subsection{Mandated Restraints}
		\subsubsection*{Constraint 1}  \label {constraint1}
		\indent The source code of the application shall be implemented 
		using the Python programming language, version \PYVER\\
		\textbf{Rationale 1}\\
		\indent The product should preserve compatibility for existing users of 
		the original implementation. The client should not be required to 
		upgrade or install new software in order to run the application.\\
		\textbf{Fit Criterion}\\
		\indent The product shall only require a Python interpreter 
		(and/or the inclusion of additional Python packages) in order to 
		operate. Exception will be made if either the entirety of the product 
		is ported to another language or if the necessary runtimes and 
		dependencies can be bundled with the product.\\\\
		
		\subsubsection*{Constraint 2} \label {constraint2}
		\indent The product shall be compatible with Windows operating systems 
		(Windows 10), Mac OSX, and Ubuntu provided that Python version \PYVER
		is compatible with the target system.\\
		\textbf{Rationale 2}\\
		\indent The client should not be required to migrate to newer (or 
		different) software or hardware in order to utilize the product.\\
		\textbf{Fit Criterion}\\
		\indent The product shall be tested on each of the specified systems by 
		the developers to ensure that it operates as expected and is fully 
		functional.\\\\
		
		\subsubsection*{Constraint 3} \label {constraint3}
		\indent The product shall be made available under the GNU GPLv3 or 
		later, along with any non-permissive terms added in accord with section 
		7 of the GNU GPLv3.\\
		\textbf {Rationale 3}\\
		\indent The original implementation was conveyed with this license and 
		as per 
		the conditions modified source versions must be licensed under the same 
		license.\\
		\textbf {Fit Criterion}\\
		\indent The product must adhere to the conditions of the GNU GPLv3 
		license.\\\\
		
		\subsubsection*{Constraint 4} \label {constraint4}
		\indent The source code for the product shall be publicly available but 
		the end product must also be deliverable as an standalone executable or 
		application for both Windows operating systems and Mac OSX operating 
		system.\\
		\textbf {Rationale 4}\\
		\indent The source code for the product must be publicly available as 
		per the conditions of the license. Users should not be required to be 
		familiar with the command-line/terminal in order to operate the 
		product.\\
		\textbf {Fit Criterion}\\
		\indent The final product should be available as a .exe executable for 
		Windows-based systems and as an application for OSX systems. The code 
		should be available in a publicly accessible repository.\\
		
		\subsection{Naming Conventions and Terminology}
		\textbf {Source code:} The form of the product that is preferred to be used
		for the making of modifications\\
		\textbf {Compatibility:} The ability of the product to operate on a given
		system. Full functionality is expected, unless otherwise specified, but
		performance, appearance, and other characteristics may vary\\
		\textbf {Implementation:} The object code, binaries, and/or executable form
		of the product, along with the source code used to create it\\
		\textbf {Product:} The entirety of the project, including but not limited to
		its source code, binaries, and documentation\\
		
		\subsection{Relevant Facts and Assumptions}
		
		\textbf{Relevant Facts}
		\vspace{-2mm}
		\begin{itemize}
			\itemsep0em
			\item The existing implementation is implemented purely in 
			the Python programming language.
			\item Mac OSX and Ubuntu come with a Python environment 
			pre-installed. The installed version may or may not be compatible 
			with 
			the product.
		\end{itemize}
		\textbf{Assumptions}
		\vspace{-2mm}
		\begin{itemize}
			\itemsep0em
			\item 	It is not feasible for the product to be tested for 
			compatibility on all of the various operating systems. As such it 
			will only be tested on the latest version of each major operating 
			system as previously specified.
			\item The product assumes that the system that it will be operated 
			on includes peripherals such as a mouse, keyboard and display.
		\end{itemize}
		
	\section{Functional Requirements}
		\subsection*{The Scope of the Work and the Product}
		\subsection{The Current Situation}
		\indent There already exists an implementation that has a similar purpose as
		that of our product, namely PySol Fan Club Edition. This implementation was
		coded entirely in Python and is an open-source software product.\\
		\subsection{The Context of the Work}
		\indent Our product shall be based upon the implementation PySol Fan Club
		Edition however it shall be redeveloped in order to satisfy the constraints
		and requirements defined in this document.\\
		\subsection{Work Partitioning}
		\subsection{Individual Product Use Cases}
		\subsection{Functional Requirements}
		\indent \indent \textbf {Functional Requirement 1} \label{freq1}\\
		\indent \indent The program will be executed on a computer compatible with
		Python version \PYVER\\
		\indent \textbf {Fit Criterion}\\
		\indent \indent The functionality of the program on the major operating
		systems - Windows 10, Ubuntu, Mac OSX - will be tested before release\\\\
		\indent \textbf {Functional Requirement 2} \label{freq2}\\
		\indent \indent The program will execute in its own graphical window\\
		\indent \textbf {Fit Criterion}\\
		\indent \indent Check that the graphical user interface is successfully
		created and fully functional on the target system(s).\\\\
		\indent \textbf {Functional Requirement 3} \label{freq3}\\
		\indent \indent The user’s game statistics will persist even after the 
		application is	terminated\\
		\indent \textbf {Fit Criterion}\\
		\indent \indent Upon re-launching the application the player statistics will
		correspond to the statistics present when the application was most recently
		terminated.\\\\
		\indent \textbf {Functional Requirement 4} \label{freq4}\\
		\indent \indent The program shall provide the user with the option to start a 
		new game\\
		\indent \textbf {Fit Criterion}\\
		\indent \indent The user interface should contain a button that when pressed 
		will trigger the start of a new game.
	
	\section{Non-functional Requirements}
		\subsection{Look and Feel Requirements}
		\begin{itemize}
			\itemsep0em
			\item The product shall represent cards with images of playing 
			cards and additional areas that can be interacted with (ex. where 
			cards can be placed) shall be outlined for the user to see.
		\end{itemize}
		\subsection{Usability and Humanity Requirements}
		\begin{itemize}
			\itemsep0em
			\item The product shall be easy for a child with basic reading and 
			computer abilities to use
		\end{itemize}
		\subsection{Performance Requirements}
		\begin{itemize}
			\itemsep0em
			\item Normal interaction actions with the game shall take no longer 
			than if an average user were playing the same game with a physical 
			deck of cards.
		\end{itemize}
		\subsection{Operational and Environmental Requirements}
		\begin{itemize}
			\itemsep0em
			\item Users will interact with the product using a mouse and 
			keyboard connected to their computer
		\end{itemize}
		\subsection{Installability Requirements}
		\begin{itemize}
			\itemsep0em
			\item The game will not need to be installed to the user's device; instead
			it will be packaged and self-contained, runnable as a portable application.
		\end{itemize}
		\subsection{Maintainability and Support Requirements}
		\begin{itemize}
			\itemsep0em
			\item The product is expected to run on Windows 10, Mac OSX, and 
			Ubuntu systems that have Python version \PYVER installed.
		\end{itemize}
		\subsection{Security Requirements}
		\begin{itemize}
			\itemsep0em
			\item The source code for the product is publicly available and as such
			anyone can download, modify, and produce modified version of the project.
			However, only the developer team is allowed to directly make changes to the 
			repository
			\item When the product is made ready for distribution, the packaged product
			shall have a checksum (ie. MD5 checksum) for verifying authenticity
		\end{itemize}
		\subsection{Cultural Requirements}
		\begin{itemize}
			\itemsep0em
			\item The product will not contain images or text that would be 
			considered offensive to residents of North America.
		\end{itemize}
		\subsection{Legal Requirements}
		\begin{itemize}
			\itemsep0em
			\item The product is licensed under the GNU GPLv3 license and must 
			conform to the license and any non-permissive terms added in 
			accordance to section 7 of the GNU GPLv3 license.
		\end{itemize}
		\subsection{Health and Safety Requirements}
		\begin{itemize}
			\itemsep0em
			\item This product requires the use of a pointing device and that the user
			views and interacts with a graphical user interface. Excessive use of this
			product may cause strain or injury including but not limited to eye strain
			and injury to arms, wrists, and/or hands including carpal tunnel syndrome.
		\end{itemize}
		
	\section{Project Issues}
		\subsection{Open Issues}
<<<<<<< HEAD
		\indent \indent The only know open issue is the possibility of the graphics library provided by the open source project being unusable. Otherwise there aren't any forcable issues because the team has permission to use the existing.\\
		\indent TPython includes a lot of built-in libraries that will be used in the implementation. One such library is Tkinter, used for the graphical portion of the product.\\
		\indent The existing implementation PySolFC is an off-the-shelf solution, and so is the project that PySolFC was based upon, the original PySol.\\
		\subsection{New Problems}
		\indent \indent Any changes to the original PySol graphics library that would make it not usable for our reimplementation would affect the work of the developers. They would then have to recreate the graphics library. The program should not require a lot of RAM when it’s running or take up too much space on the user’s system. The project should not over power the servers or over consume power.\\
		\subsection{Tasks}
		\indent \indent The team plan on completing this project by completely problem statement and requirements documents before beginning programming. With these documents the team will have a strong understanding of what needs to done. From then on developers will start creating the program using the Model View Controller method and referring to the open sourced project when needed. To manage the project we’ll be using version control with git. \\


		\subsection{Migration to the New Product}
		\indent \indent No real requirements for migration as we plan to deliver as a standalone executable. Packaging (bundling) as a standalone executable allows the product to be self-contained with minimal external dependencies.\\
		\indent However, the existing implementation if installed adds its modules to a python directory for 3rd party packages. This will need to be reversed and/or the new files written to the same directory.\\
=======
		\begin{itemize}
		\itemsep0em	
			\item The main open issue is the possibility of the graphics extension
			provided by the open source project being incompatible. Seeing as the team
			has permission to use and modify the existing implementation, many of the
			issues a new project would face have been accounted for
			\item Python includes a lot of built-in libraries that will be used in the
			 implementation. One such library is Tkinter, used for the graphical portion
			of the product
			\item The existing implementation PySolFC is an off-the-shelf solution,
			and so is the project that PySolFC was based upon, the original PySol
		\end{itemize}
		\subsection{New Problems}
		\begin{itemize}
		\itemsep0em	
			\item Any changes to the original Tkinter graphics library or the 
			custom interfaces that PySolFC provides to the library could make it
			incompatible with our reimplementation and would thusly affect the work of
			our developer team. It would require the adaptation of our code to the new
			API which could introduce problems
			\item The program should not require a lot of RAM when it’s running or
			have high disk usage or requirements of the user’s system. Overall the
			project should not have overly high resource consumption on the target system\\
		\end{itemize}
		\subsection{Tasks}
			\indent The team plan on completing this project by completing the problem
			statement and requirements documents before beginning programming. With these
			documents the team will have a strong understanding of what needs to done.
			From then onwards developers will work on implementing the program using
			the Model View Controller software architecture and referring to the existing
			project if and when needed. To manage the project we’ll be using version
			control via git.\\


		\subsection{Migration to the New Product}
			\indent There are no major requirements for migration as we intend to
			distribute the program as a standalone executable (while making the source
			available to interested parties). As such the product will be able to coexist
			with the existing products PySol and PySolFC. Furthermore, packaging
			or bundling as an executable (or application) allows the product to be
			self-contained with minimal external dependencies.\\
			\indent Yhe existing implementation however, if installed, adds its modules
			to the python directory for 3rd party packages. Facilitating its removal
			would require us to reverse that modification and by removing or updating
			the specified files\\
>>>>>>> ebf8b3a7fbb710505649cb0593de42bbec0f1382

		\subsection{Risks}
		\begin{itemize}
			\itemsep0em
			\item Excessive pressure due to time constraints for deliverables
			\item Inaccurate quantification of objectives
			\item Inadequate testing of graphical user interface
		\end{itemize}

		\subsection{Costs}
<<<<<<< HEAD
		\indent \indent The product being developed is open-source as is the implementation it is based upon. To follow and propagate the concept of open-source software, only freely available software shall be used in the development process. \\
		\indent The primary and only significant cost of the project is the time and effort spent by its developers. The cost is anticipated to be 5 hours or more of time spent per person per week for the duration of September-December 2016.\\

		\subsection{User Documentation and Training}
		\indent \indent In-person training will not be required and therefore will not be provided for this product. However, user documentation will be included with the packaged product. A specifications document including APIs, modular decomposition, and also build & modification instructions shall accompany the source for the product.\\
 		\indent The user documentation shall provide enough guidance that a user with no previous exposure to the product shall after reading be able to achieve basic objectives. These objectives include but are not limited to navigating the interface, starting a new game, and accessing the demo feature.\\
 		\indent The user documentation shall include a section (or multiple sections) that display the rules of the available games.
=======
			\indent The product being developed is open-source as is the implementation
			it is based upon. To follow and propagate the concept of open-source
			software, only freely available software shall be used in the development
			process.\\
			\indent The primary and only significant cost of the project to date is the
			time and effort spent by its developers. The cost is anticipated to be around
			5 hours or more of time spent per person per week for the duration of
			September-December 2016.\\

		\subsection{User Documentation and Training}
			\indent In-person training will not be required and therefore will not be
			provided for this product. However, user documentation will be included with
			the packaged product. A specifications document including APIs, modular
			decomposition, and also build and modification instructions shall accompany
			the source for the product.\\
	 		\indent The user documentation shall provide enough guidance that a user
			with no previous exposure to the product shall after reading be able to
			achieve basic objectives. These objectives include but are not limited to
			navigating the interface, starting a new game, and accessing the demo
			feature.\\
	 		\indent The user documentation shall include a section (or multiple
			sections) that display the rules of the available games.
>>>>>>> ebf8b3a7fbb710505649cb0593de42bbec0f1382

		\subsection{Waiting Room}		
		\indent Some possible requirements we could address if time permits:\\
		\begin{itemize}
			\itemsep0em
			\item Create more games (version 1.1)\\
			By creating more games users will have more options and be less likely to
			become bored or disinterested
			\item Improve the graphics (version 1.1) \\
			By improving the graphics, the program will avoid the risk of seeming
			outdated or becoming obsolete
			\item Create a leaderboard (version 1.2)\\
			By creating a leaderboard, it will keep people more engaged and bring out
			their competitiveness which will motivate them to use the product more
		\end{itemize}		
		\subsection{Ideas for Solutions}
		
	\newpage
	\section{Appendix}
	\subsection{Symbolic Parameters}
		

\end{document}

