\documentclass{article}

\usepackage[letterpaper, margin=1in]{geometry}
\usepackage{setspace}
\usepackage{tabularx}
\usepackage{booktabs}
\usepackage{setspace}
\usepackage{hyperref}
\usepackage{color}


\title{SE 3XA3: Development Plan\\PyCards}

\author{Team 2 \\ \\ Aravi Premachandran premaa \\ Michael Lee leemr2
\\ Nikhil Patel patelna2 }

\date{}

\begin{document}

\maketitle

\begin{table}[hp]
\caption{Revision History} \label{TblRevisionHistory}
\begin{tabularx}{\textwidth}{llX}
\toprule
\textbf{Date} & \textbf{Developer(s)}& \textbf{Change}\\
\midrule
Sep 30, 2016 & Aravi Premachandran, Michael Lee, Nikhil Patel
& Initial Revision\\ 
Nov 26, 2016 & Nikhil Patel
& Added description of how labels will be used as part of Git Workflow\\ 
Nov 26, 2016 & Nikhil Patel
& Added to description of how to the Proof of Concept Demonstration
shall be conducted\newline Added Project Review section heading\\ 
\bottomrule
\end{tabularx}
\end{table}

\section{Team Meeting Plan}

\indent Team meetings are an essential part of the development process. It is
important that the members of this group approach this project in a structured
fashion and remain organized throughout. Meetings will be held during the latter
portion of the regularly scheduled lab sections (time permitting) or immediately
afterwards. In addition to the above listed times (or in lieu of in the case of
schedule conflicts), informal meetings shall be conducted via e-communication
using either of the Facebook and Skype platforms.
\newline \indent Meetings will be scheduled on a weekly basis to ensure there
is no backlog of issues needing resolving, to keep the project progressing on
schedule, and as a general debrief on the past week and for the coming week(s).
Meetings will generally start with a review of the past week(s) progress,
including but not limited to deliverables completed and deadlines met. Then any
issues or concerns that have arisen or are ongoing shall be raised by members
of the group and addressed. Finally, meetings shall conclude with a summary of
progress, a reiteration of changes made and/or issues resolved during the
course of the meeting, and an overview of upcoming tasks and deliverables.

\section{Team Communication Plan}

\indent The primary means of communication for the members of this team shall
be through the instant messaging and conferencing services available using
Facebook and Skype. These platforms are both convenient and also free to use -
as such they are ideal for a project of this scale. Impersonal and routine
issues, comments, and conversation shall be done using Facebook. Any personal or
heavily involving issues or communications shall be done using Facebook if
satisfactory to all parties or using Skype if teleconferencing is deemed
necessary for efficient communication and resolution.

\newpage

\section{Team member roles}

\noindent Leader: Nikhil Patel \newline \noindent Experts: \vspace{-2mm}
\begin{itemize} \itemsep0em \item Documentation: Aravi Premachandran \item Git:
Nikhil Patel \item LaTeX: Michael Lee \end{itemize}

\section{Git workflow plan}

\indent Our team will be implementing the Feature Branch Workflow using Git.
The reasoning behind using the feature branch workflow is that the master branch
should always contain stable, correct code. Development or modification of a new
or old feature will be done within a dedicated branch, and when completed a pull
request will be filed before merging with master. Using the feature branch
workflow isolates the development of features from the core codebase making it
easier for multiple members to work on the same or different features
simultaneously while keeping the most recent working version of the software
intact.\newline
\indent \textcolor{red}{Gitlab also has an integrated issue tracking system that we will be
using throughout the course of the project. Issues are a way to manage and keep
track of tasks, features, bugs and much more. The main focus of issues is on
collaboration. We intend to organize the issues we create using labels and
issue templates.  Labels function similar to keyword tags for the issue, and
can be used to identify and also filter issues for better organization. Issue
templates will used to provide a general format when presenting an issue to
help ensure that issues contain enough relevant information to understand and
address them effectively.}


\section{Proof of Concept Demonstration Plan}

\indent Throughout the software development process there are risks that our
team will face. Some of the risks that we may face are difficulty testing the
application, the software product's dependence on the user having Python
installed on the host system, and also adapting the product to use automated
build and testing tools.
\newline \indent The software product implements a graphical user
interface which makes it difficult to test its functionality. It will not be
practical or even feasible to use automated testing on the user interface. As
such, we will be forced to use behavioural testing ie. running the application
and validating its response to various inputs from the user.
\newline \indent Another risk we may encounter is difficulty in packaging the
product as a portable and standalone application. The existing implementation
relies on the user having Python installed on the host system - to overcome
this we will need to perform research as to if and how the Python environment
can be bundled with the application, and if it is legal to do so.
\newline \indent The next risk that we have to address is the desire for an
automated building and testing environment. If different users or developers
attempt to build and deploy the source in an unintended manner it may lead to
unexpected behaviour or errors. To address this need we will look into using a
build tool or writing a custom build script (ie. makefile). Having a defined
build script and/or using a build tool makes the build and also deployment
process consistent and also convenient for both users and developers. \newline
\indent \textcolor{red}{The proof of concept demonstration will be a
demonstration of a minimal user interface implemented in Python and using 
the Tkinter and ttk packages, as the design and implementation of the graphical
portion of the system is anticipated to be one of the primary challenges. In
addition to the implementation described above, the team will demonstrate their
ability to compile, package and launch the program as a single executable file.
Finally, the team shall create and show the successful execution of a subset of
automated tests (to demonstrate domain knowledge, competence and feasibility)
of conducting automated testing using the unittest.py module that comes with a
default Python installation.}

\section{Technology}

\indent The software product shall be redeveloped in the
same programming language as the existing implementation, Python 2.7.x . The IDE
of choice will be the simplistic IDLE editor though any execution of the
implementation shall be done using the command-line (shell). The pydoc
documentation tool will tentatively be used for document generation, and
pybuilder will tentatively be used both as a build tool and for testing.

\section{Coding Style} 

\indent Python is a dynamic language in that variables are not bound to a
type, only the value(s) stored in it is. Python also poorly implements the
principle of least privilege. As such, it is especially important to follow a
consistent coding style in order to reduce the likelihood of making simple but
easy-to-miss errors such as type errors. We plan on adhering to a combination 
of the\href{https://google.github.io/styleguide/pyguide.html}
{Google Python Style Guide} and the Python style guide defined in the 
\href{https://www.python.org/dev/peps/pep-0008/} {Python Developer's Guide}.
When reviewing our code, we will run the tool pylint over our code to help
enforce it. Pylint is a tool that finds bugs and style issues in Python source.

\section{Project Schedule} 

\noindent The Gantt Project schedule for this project can be found in the
Project Schedule folder.

\section{Project Review} 



\end{document}
