\documentclass{article}

\begin{document}
	\begin{center}
		\section*{PyCards : Problem Statement}
		Group 2 : Aravi Premachandran, Michael Lee, Nikhil Patel
	\end{center}
	
	\section{Introduction}
	A common problem that plagues many people is the need to pass time, 
	especially when lacking internet access. Card and puzzle games such as 
	FreeCell, Hearts, and Solitaire are designed to meet this need and are 
	traditionally bundled with other software such as operating systems. 
	However, those versions usually offer limited choice in games and in some 
	cases have or may become populated with ads and obscure functionality. As 
	such there is a need for open-source software products that offer a variety 
	of solitaire games within a single standalone application.
	
	\section{Objective}
	There are numerous applications that exist to meet a similar need, but one 
	of the most distinguishing characteristics of PySol is the vast variety of 
	distinct solitaire games offered. This is significant because the average 
	user is likely to get bored if limited to only a single or a few games all 
	with similar mechanics. There is a need for an application that offers a 
	large variety of solitaire games and provides a diversified, engaging and 
	entertaining experience.
	
	\section{Context}
	The stakeholders for this project would include computer game players of 
	all ages. Children and young adults are likely to be the dominant 
	demographic.
	\newline
	\newline
	The software is an open-source desktop application developed purely in 
	Python. It is compatible with Windows operating systems and Unix-based 
	operating systems such as Mac OSX. The software is intended to be used on 
	any typical laptop or desktop computer. The software lacks documentation 
	and as such the scope of this program will be to re-implement the existing 
	software and create and provide formal documentation.
	
\end{document}